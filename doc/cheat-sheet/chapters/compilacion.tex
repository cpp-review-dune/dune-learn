Para compilar un programa en C++ que emplee las cabeceras de DUNE, es
necesario construir una carpeta temporal de construcción que
llamaremos \verb|build|, que se utiliza para construir allí los
aspectos necesarios para la compilación del proyecto, y aparecerá
después de compilar el archivo ejecutable.
Una vez creada la carpeta, es necesario utilizar el comando
\verb|cmake|.
En éste caso se indica a cmake que utilice como fuente
(``S'' de source) la carpeta \verb|dune-prueba| y que la construcción
de la compilación se va a guardar en la carpeta \verb|build|, por eso
se escribe (``-B'' build).

\begin{verbatim}
gitpod/workspace $ cmake -S dune-prueba -B build
\end{verbatim}

En caso de que se quiera compilar en la misma carpeta se debe
utilizar el comando:
\begin{verbatim}
gitpod/workspace $ cmake -S . -B build
\end{verbatim}

Luego de hacer el proceso, si hacemos una lista utilizando el comando
\verb|ls|, obtendremos dos carpetas, a saber:

\begin{verbatim}
gitpod/workspace $ ls
gitpod/workspace $ build dune-prueba
\end{verbatim}

Es necesario aclarar, que nuestro código estará en la carpeta
\verb|dune-prueba|, mientras que el ejecutable y todos los archivos
necesarios para la compilación están en la carpeta \verb|build|.
A continuación, ingresamos a la carpeta \verb|build| y usamos el
comando \verb|make|:

\begin{verbatim}
gitpod/workspace $ cd build
gitpod/workspace/build $ make
\end{verbatim}

Con lo que se iniciará el proceso de generación del archivo
ejecutable, que estará ubicado en éste caso, en la dirección
siguiente y se debe ejecutar de la siguiente forma:

\begin{verbatim}
gitpod/workspace/build/src $ ./dune-prueba
\end{verbatim}

El programa y su salida se presentan a continuación:
\begin{listing}[ht!]
	\inputminted{bash}{dune-learn-1.txt}
\end{listing}

\url{https://man.archlinux.org/man/cmake.1}