\subsection[Primer Proyecto]{Primer proyecto en Dune}

Para iniciar la ejecución, una vez hemos ingresado a la plataforma y
estamos en la nube, debemos ver una imagen similar a la siguiente:

\begin{center}\label{fig:ingreso}
	\includegraphics[scale=0.3]{PantallaIngreso.png}
\end{center}
Como se puede apreciar, es muy similar a Vscode.  Para hacer mi primer proyecto, en la terminal debo escribir:
\textcolor{green}{gitpod} \textcolor{blue}{/workspace} \$ duneproject
\begin{verbatim}

== Dune project/module generator ==

duneproject will assist you in the creation of a new Dune application.
During this process a new directory with the name of your project will be
created. This directory will hold all configuration and Makefiles and a
simple example application.

1) Name of your new Project? (e.g.: dune-grid): 
\end{verbatim}
En ésta parte, debo completar la información de mi proyecto, el nombre vamos a poner por ejemplo \verb|dune-prueba|, luego
\begin{verbatim}
	2) Which modules should this module depend on?
   The following modules have been found:
   dune-common dune-geometry dune-uggrid dune-grid dune-localfunctions dune-istl 
   dune-typetree dune-functions dune-alugrid dune-pdelab dune-fem dune-learn 
   Enter space-separated list:
\end{verbatim}
En éste caso, vamos a usar \verb|dune-pdelab|, luego pregunta:
\begin{verbatim}
	3) Project/Module version? 
\end{verbatim}
Por ejemplo puede ser $01$, luego nos pide el email:
\begin{verbatim}
	4) Maintainer's email address? maintainer@unal.edu.co 
\end{verbatim}
Finalmente, nos pregunta si los datos están correctos:
\begin{verbatim}
	creating Project "dune-prueba", version 01 
which depends on "dune-pdelab"
with maintainer "maintainer@unal.edu.co"
Is this information correct? [y/N] y
\end{verbatim}
Si los datos son correctos, asignamos ''y'', de lo contrario ''N'', y finalmente enter para iniciar la configuración. Una vez iniciada la configuración,
se crea una carpeta con el nombre \verb|dune-prueba|, que fue el nombre del proyecto que asignamos.  Una vez que
el proyecto se ha construido, podemos utilizar el comando \begin{verbatim}
	gitpod /workspace $ ls
\end{verbatim}
Dentro del listado debe aparecer la carpeta \textcolor{blue}{dune-prueba}, que si explora dentro de ella, encontrará
varios archivos y carpetas.  Se puede dirigir a la siguiente dirección: {\begin{verbatim}
	cd /workspace/dune-prueba/src $ 
\end{verbatim}}Cuando liste, encontrará dos archivos \begin{verbatim}
	CMakeLists.txt  dune-prueba.cc
\end{verbatim}
En el primer archivo, usted encuentra lo siguiente:
\begin{verbatim}
	add_executable("dune-prueba" dune-prueba.cc)
	target_link_dune_default_libraries("dune-prueba")
\end{verbatim}
Significa que se ha creado un código fuente, que se llama ''dune-prueba.cc'', tiene el mismo nombre del proyecto que
creamos, y en el que está escrito nuestro primer programa, el ''Hola mundo de DUNE''.

A continuación se puede apreciar el contenido completo del primer programa :
\begin{listing}
	\inputminted{cpp}{../../src/dune-learn.cc}
\end{listing}
\subsection{Compilación}
Para compilar el programa, se va a generar una carpeta \textit{build},
que se construye a partir del comando \textit{cmake}, ésta carpeta
se utiliza para construir allí los aspectos necesarios para la compilación del proyecto, y aparecerá después
de utilizar la siguiente instrucción.
\begin{verbatim}
	gitpod /workspace $ cmake -S dune-prueba -B build
\end{verbatim}
En éste caso se indica a cmake que utilice como fuente (''S'' de source) la carpeta \verb|dune-prueba| y
que la construcción de la compilación se va a guardar en la carpeta \verb|build|, por eso se escribe (''-B'' build),
que significa construya en la carpeta \textit{build}.
En caso de que se quiera compilar en la misma carpeta se debe utilizar el comando:
\begin{verbatim}
	gitpod /workspace $ cmake -S . -B build
\end{verbatim}
Luego de hacer el proceso, si hacemos una lista utilizando el comando \verb|ls|, obtendremos dos carpetas, a saber:
\begin{verbatim}
	gitpod /workspace $ ls
	gitpod /workspace $ build dune-prueba
\end{verbatim}
Es necesario aclarar, que nuestro código estará en la carpeta \verb|dune-prueba|, mientras que el ejecutable y
todos los archivos necesarios para la compilación están en la carpeta \verb|build|.  A continuación, ingresamos
a la carpeta \verb|build| y usamos el comando \verb|make|:
\begin{verbatim}
	gitpod /workspace $ cd build
	gitpod /workspace/build $ make
\end{verbatim}
Con lo que se iniciará el proceso de generación del archivo ejecutable, que estará ubicado en éste caso, en
la dirección siguiente y se debe ejecutar de la siguiente forma:
\begin{verbatim}
	gitpod /workspace/build/src $ ./dune-prueba
\end{verbatim}
La salida después de ejecutar el código, se muestra a continuación:
\begin{listing}[ht!]
	\inputminted{bash}{dune-learn-1.txt}
\end{listing}

\immediate\write18{./run.sh}
