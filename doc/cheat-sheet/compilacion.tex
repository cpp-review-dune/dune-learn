Para compilar el programa, vamos a crear la  carpeta build, haciendo lo siguiente: 
\begin{verbatim}
	gitpod /workspace $ mkdir build
\end{verbatim}
Build, se utiliza para construir allí los aspectos necesarios para la compilación del proyecto, y aparecerá después 
de compilar el archivo ejecutable.  Una vez creada la carpeta, es necesario utilizar el comando \textbf{Cmake}.  En éste caso se 
indica a cmake que utilice como fuente (''S'' de source) la carpeta \textbf{dune-prueba} y que la construcción de la 
compilación se va a guardar en la carpeta \textbf{build}, por eso se escribe (''-B'' build). 
\begin{verbatim}
	gitpod /workspace $ cmake -S dune-prueba -B build
\end{verbatim}
En caso de que se quiera compilar en la misma carpeta se debe utilizar el comando: 
\begin{verbatim}
	gitpod /workspace $ cmake -S . -B build
\end{verbatim}
Luego de hacer el proceso, si hacemos una lista utilizando el comando \textbf{ls}, obtendremos dos carpetas, a saber:
\begin{verbatim}
	gitpod /workspace $ ls
	gitpod /workspace $ build dune-prueba
\end{verbatim}
Es necesario aclarar, que nuestro código estará en la carpeta \textbf{dune-prueba}, mientras que el ejecutable y 
todos los archivos necesarios para la compilación están en la carpeta \textbf{build}.  A continuación, ingresamos 
a la carpeta \textbf{build} y usamos el comando \textbf{make}:
\begin{verbatim}
	gitpod /workspace $ cd build
	gitpod /workspace/build $ make
\end{verbatim}
Con lo que se iniciará el proceso de generación del archivo ejecutable, que estará ubicado en éste caso, en 
la dirección siguiente y se debe ejecutar de la siguiente forma:
\begin{verbatim}
	gitpod /workspace/build/src $ ./dune-prueba
\end{verbatim}

El programa y su salida se presentan a continuación:
\begin{listing}[ht!]
	\inputminted{bash}{dune-learn-1.txt}
\end{listing}