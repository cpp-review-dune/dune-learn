Éste manual pretende introducir a los usuarios en el manejo del software \textbf{Dune}, 
que se puede ubicar en la dirección 
\url{https://dune-project.org/}, software de simulación numérica DUNE (Distributed and Unified Numerics Environment), 
proyecto de código abierto que proporciona una infraestructura flexible y 
modular para la solución de ecuaciones diferenciales parciales.

\section{Instalación}

Para la correr el software DUNE, vamos a utilizar el repositorio \url{https://github.com/cpp-review-dune/dune-learn},
diseñado para funcionar en la nube, utilizando las ventajas que nos ofrece \url{https://www.gitpod.io/}, en él 
se utiliza una imagen docker prediseñada, que contiene las dependencias y repositorios necesarios para su funcionamiento.
\footnote{La extensión de gitpod puede ser instalada dependiendo del browser
que se utilice}Ver la figura \ref{fig:github01}:
\begin{center}\label{fig:github01}
    \includegraphics*[scale=0.3]{cppreview-learn.png}
\end{center}

En la sección de C++, se presentará \textbf{dunepdelab}, que fue estudiada en el curso presentado en el año 
$2022$, de manera remota y cuya link está en: \url{https://dune-pdelab-course.readthedocs.io/en/latest/intro.html},
y en la sección de Python, utilizaremos el Jupyter Notebook para introducir el uso de \textbf{dunefem}.