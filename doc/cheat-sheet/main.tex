\documentclass{scrarticle}
\usepackage{minted}
\usepackage{hyperref}
\usepackage{graphicx}
\usepackage{color}

\begin{document}
\section{Introducción}
Éste manual pretende introducir a los usuarios en el manejo del software \textbf{Dune}, 
que se puede ubicar en la dirección 
\url{https://dune-project.org/}, software de simulación numérica DUNE (Distributed and Unified Numerics Environment), 
proyecto de código abierto que proporciona una infraestructura flexible y 
modular para la solución de ecuaciones diferenciales parciales.

\section{Instalación}

Para la correr el software DUNE, vamos a utilizar el repositorio \url{https://github.com/cpp-review-dune/dune-learn},
diseñado para funcionar en la nube, utilizando las ventajas que nos ofrece \url{https://www.gitpod.io/}, en el 
que su puede utilizar una imagen docker, que contiene las dependencias y repositorios necesarios para su funcionamiento.

En la sección de C++, se presentará \textbf{dunepdelab}, que fue estudiada en el curso presentado en el año 
$2022$, de manera remota y cuya link está en: \url{https://dune-pdelab-course.readthedocs.io/en/latest/intro.html},
y en la sección de Python, utilizaremos el Jupyter Notebook para introducir el uso de \textbf{dunefem}.
\section{C++}
\subsection[Prier Proyecto]{Primer proyecto en Dune}
Para iniciar la ejecución, una vez hemos ingresado a la plataforma y estamos en la nube, debemos ver una imagen similar 
a la siguiente:
\begin{center}\label{fig:ingreso}
	\includegraphics[scale=0.3]{PantallaIngreso.png}
\end{center}
Es muy similar a Vscode, cómo se puede apreciar.  Para hacer mi primer proyecto, en la terminal debo escribir:
\textcolor{green}{gitpod} \textcolor{blue}{/workspace} \$ duneproject 
\begin{verbatim}

== Dune project/module generator ==

duneproject will assist you in the creation of a new Dune application.
During this process a new directory with the name of your project will be
created. This directory will hold all configuration and Makefiles and a
simple example application.

1) Name of your new Project? (e.g.: dune-grid): 
\end{verbatim}
En ésta parte, debo completar la información de mi proyecto, el nombre vamos a poner por ejemplo \textbf{dune-prueba}, luego
\begin{verbatim}
	2) Which modules should this module depend on?
   The following modules have been found:
   dune-common dune-geometry dune-uggrid dune-grid dune-localfunctions dune-istl 
   dune-typetree dune-functions dune-alugrid dune-pdelab dune-fem dune-learn 
   Enter space-separated list:
\end{verbatim}
En éste caso, vamos a usar \textbf{dune-pdelab}, luego pregunta:
\begin{verbatim}
	3) Project/Module version? 
\end{verbatim}
Por ejemplo puede ser $01$, luego nos pide el email:
\begin{verbatim}
	4) Maintainer's email address? maintainer@unal.edu.co 
\end{verbatim}
Finalmente, nos pregunta si los datos están correctos:
\begin{verbatim}
	creating Project "dune-prueba", version 01 
which depends on "dune-pdelab"
with maintainer "maintainer@unal.edu.co"
Is this information correct? [y/N] y
\end{verbatim}
Si los datos son correctos, asignamos ''y'', de lo contrario ''N'', y finalmente enter para iniciar la configuración. Una vez iniciada la configuración,
se crea una carpeta con el nombre \textbf{dune-prueba}, que fue el nombre del proyecto que asignamos.  Una vez que 
el proyecto se ha construido, podemos utilizar el comando \begin{verbatim}
	gitpod /workspace $ ls
\end{verbatim}
Dentro del listado debe aparecer la carpeta \textcolor{blue}{dune-prueba}, que si explora dentro de ella, encontrará 
varios archivos y carpetas.  Se puede dirigir a la siguiente dirección: {\begin{verbatim}
	cd /workspace/dune-prueba/src $ 
\end{verbatim}}Cuando liste, encontrará dos archivos \begin{verbatim}
	CMakeLists.txt  dune-prueba.cc
\end{verbatim}
En el primer archivo, usted encuentra lo siguiente:
\begin{verbatim}
	add_executable("dune-prueba" dune-prueba.cc)
	target_link_dune_default_libraries("dune-prueba")

\end{verbatim} 
Significa que se ha creado un código fuente, que se llama ''dune-prueba.cc'', tiene el mismo nombre del proyecto que 
creamos, y en el que está escrito nuestro primer programa, el ''Hola mundo de DUNE''.  A continuación se puede apreciar 
el contenido completo del primer programa :

\begin{listing}[ht!]
	\inputminted{cpp}{../../src/dune-learn.cc}
\end{listing}

\immediate\write18{./run.sh}
\subsection{Compilación}
Para compilar el programa, vamos a crear una carpeta 
\begin{verbatim}
	gitpod /workspace $ mkdir build
\end{verbatim}
Build, se utiliza para construir allí los aspectos necesarios para la compilación del proyecto, y aparecerá después 
de compilar el archivo ejecutable.  Una vez creada la carpeta, es necesario utilizar el comando \textbf{Cmake}.  En éste caso se 
indica a cmake que utilice como fuente (''S'' de source) la carpeta \textbf{dune-prueba} y que la construcción de la 
compilación se va a guardar en la carpeta \textbf{build}, por eso se escribe (''-B'' build) 
\begin{verbatim}
	gitpod /workspace $ cmake -S dune-prueba -B build
\end{verbatim}
Luego de hacer el proceso, si hacemos una lista, tendremos:
\begin{verbatim}
	gitpod /workspace $ build dune-prueba
\end{verbatim}
Es necesario aclarar, que nuestro código estará en la carpeta \textbf{dune-prueba}, mientras que el ejecutable y 
todos los archivos necesarios para la compilación están en la carpeta \textbf{build}.  A continuación, ingresamos 
a la carpeta \textbf{build} y usamos el comando \textbf{make}:
\begin{verbatim}
	gitpod /workspace/build $ make
\end{verbatim}
Con lo que se iniciará el proceso de generación del archivo ejecutable, que estará ubicado en éste caso, en 
la dirección y falta ejecutarlo de la siguiente forma:
\begin{verbatim}
	gitpod /workspace/build/src $ ./dune-prueba
\end{verbatim}
\begin{listing}[ht!]
	\inputminted{bash}{dune-learn-1.txt}
\end{listing}
\newpage
\section{Python}

\end{document}